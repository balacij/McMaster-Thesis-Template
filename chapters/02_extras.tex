\chapter{Extras}
\label{chap:extras}

\begin{writingdirectives}

    \item What macros do I want the reader to know about?
    
\end{writingdirectives}

\section{Writing Directives}
\label{chap:extras:sec:writing-directives}

I enjoy writing directives (mostly questions) to navigate what I should be
writing about in each chapter. You can do this using:

\begin{pseudocode}{latex}{exWD}{Example Writing Directives}
\begin{writingdirectives}
    \item What macros do I want the reader to know about?
\end{writingdirectives}
\end{pseudocode}

Personally, I put them at the top of chapter files, just after chapter
declarations.

\section{Acronyms}
\label{chap:extras:sec:acronyms}

I used a lot of acronyms in my thesis, and I wanted to add a glossary to the
front matter. This is compulsory for McMaster theses, but you may remove it if
you don't need it. I used the \texttt{acro} package, feel free to read their
\porthref{documentation}{https://ctan.org/pkg/acro?lang=en}. For example, I
might want to write \acf{html} (written using \inlineCode{latex}{\acf{html}}) or
just \acs{html} (written using \inlineCode{latex}{\acs{html}}). To define them,
you can find examples in the \texttt{acronyms.tex} file.

\section{\acsp{href}}
\label{chap:extras:sec:hrefs}

For \acsp{pdf}, we have (at least) 2 ways of viewing them: on our computers, and
printed out on paper. If you choose to view through your computer, reading links
(as they are linked in this example, inlined everywhere with ``clickable''
links) is fine. However, if you choose to read it on printed paper, you will
find trouble clicking on those same links. To mitigate this issue, I built the
``porthref'' macro (see \texttt{macros.tex} for the definition) to build links
that appear as clickable text when ``compiling for computer-focused reading,''
and adds links to footnotes when ``compiling for printing-focused reading.''
There is an option (\texttt{compilingforprinting}) in the \texttt{manifest.tex}
file that controls whether \acs{pdf} builds should be done for computers or for
printers. For example, by default, \porthref{McMaster}{https://www.mcmaster.ca/}
is made with clickable functionaity, but if you change the \texttt{manifest.tex}
option as mentioned, then you will see the link in a footnote (try it out!).


\begin{pseudocode}{latex}{exPHref}{Example Portable HREF}
\porthref{McMaster}{https://www.mcmaster.ca/}
\end{pseudocode}

\section{Code Snippets}
\label{chap:extras:sec:code-snippets}

\section{TODOs}
\label{chap:extras:sec:todos}
