\chapter{Introduction}
\label{chap:introduction}

Congratulations! If you're seeing this, it means you've managed to compile the
\acs{pdf}, which also means you can get started on typesetting your
thesis\footnote{Or report or \ldots{}}.

This template is adapted from my
\porthref{thesis}{https://github.com/JacquesCarette/Drasil/tree/master/People/Jason/thesis}.
If you'd like to see an example of this template in practice, please feel free
to use my thesis as an example.

\section{Template Organization}
\label{chap:introduction:sec:template-organization}

I've broken up the template according to my preferred organization: chapters in
separate files, various kinds of assets (images, tables, code snippets, macros,
etc.) in separate files, etc. The split is approximately according to
\refOrganizationTable{}.

\organizationTable{}

\section{Writing Tips}
\label{chap:introduction:sec:writing-tips}

When drafting chapters, I:

\begin{enumerate}
      \item wrote ``writing directives'' for each chapter to understand what I
            need to write about (see \Cref{chap:extras:sec:writing-directives}),
      \item wrote ``todo'' notes for tedious things that I might want to do
            later (such as citations, figures, code snippets, etc., see
            \Cref{chap:extras:sec:todos}), and
      \item regularly built my thesis using \inlineCode{console}{make debug} to
            make sure that whatever I wrote didn't break the \LaTeX{} code.
\end{enumerate}

For workflow recommendations, you should speak with your supervisor as they
might prefer you work in a specific way with them.

\section{Development Recommendations}
\label{chap:introduction:sec:development-recommendations}

Other than the basic tools I used for this template, I enjoyed using the
following tools while writing my thesis:

\begin{enumerate}
    \item \porthref{VS Codium}{https://vscodium.com/}/\porthref{VS
              Code}{https://code.visualstudio.com/}\footnote{I prefer VS Codium
              simply because I prefer libre software.} with the following
          extensions:
          \begin{enumerate}
              \item \porthref{\LaTeX{}
                        Workshop}{https://open-vsx.org/extension/James-Yu/latex-workshop},
                    for \LaTeX{} syntax highlighting, code formatting (this
                    is highly recommended), and code completion,
              \item \porthref{LTeX - LanguageTool grammar/spell
                        checking}{https://open-vsx.org/extension/valentjn/vscode-ltex}, for
                    grammar checking using
                    \porthref{LangaugeTool}{https://languagetool.org/}, and
              \item \porthref{Todo
                        Tree}{https://open-vsx.org/extension/Gruntfuggly/todo-tree}, for
                    quickly listing all of my TODO notes in my \acs{ide} (in addition to
                    the list at the top of the \acs{pdf}).
          \end{enumerate}
    \item \porthref{texcount}{https://ctan.org/pkg/texcount?lang=en} (which
    should come with your \LaTeX{} installation) to quickly check the word count
    of individual \LaTeX{} files, and
    \item \porthref{Zotero}{https://www.zotero.org/} for collecting my
    references and quickly exporting bib entries that I could use.
\end{enumerate}

In particular, when writing, I found it particularly helpful to use VS Code's
``Zen Mode'' (to see your keybind, press \texttt{CTRL+ALT+P} and search for
``Zen''), which enters a stripped-down full-screen version of the current
working file, keeping your eyes purely focused on the document in front of you.
Being comfortable with the keybinds is particularly helpful for working
effectively in this setup. For example, I found the following\footnote{If you're
not using Linux, I cannot guarantee that these will be the same for you, so you
should use \texttt{CTRL+ALT+P} to look for your appropriate bound keybinds.} to
be helpful: \texttt{CTRL+TAB} and \texttt{CTRL+SHIFT+TAB} to scroll between open
files, \texttt{CTRL+P} to quickly open up recent files, \texttt{CTRL+ALT+P} to
run commands you forgot the keybind for, \texttt{CTRL+O} to open up files out of
the current working directory.

While writing, I enjoyed:

\begin{enumerate}
    \item using ``TODO'' notes\eztodo{Such as this one, but check out
          \Cref{chap:extras:sec:todos} for more options.} to collect notes that
          I would want to do later,
    \item formatting the \LaTeX{} code to make it easier to read (the \LaTeX{}
          Workshop plugin has functionality for this),
    \item breaking the non-textual content into separate files and
          ``include''-ing them in the \LaTeX{} code so that they didn't cause
          large visual interruptions,
    \item using git to version control copies of my thesis, chapters, etc.,
    \item using \porthref{TikZ}{https://tikz.net/} and
          \porthref{draw.io/diagrams.net}{https://www.diagrams.net/} to build
          graphics and diagrams, and
    \item building the thesis often using \inlineCode{console}{make debug} to
          quickly debug issues in the written code.
\end{enumerate}

\section{Troubleshooting}
\label{chap:introduction:sec:troubleshooting}

``StackOverflow'' is a great area to look for solutions to common \LaTeX{}
issues. Otherwise, feel free to use create a ticket or sending an email to me.
